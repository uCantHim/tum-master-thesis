\chapter{Conclusion}

This thesis detailed the design and implementation of \textit{Focaccia}, an automatic emulator verification system. It
tests the program state of an emulator against corresponding oracle-generated states to verify whether instructions act
on it correctly. Doing this, it verifies the execution of entire programs.

Focaccia is easy to use and integrate into an emulator's development process. It has several safeguards against
real-world environment imperfections and data corruption scenarios to prevent false positives, maximizing its
reliability. Furthermore, it scales the quality of its calculations with the quality of its inputs.

The downside to our symbolic execution-based oracle is that Focaccia's ability to detect errors in the tested emulator
depends exclusively on the quality of the symbolic execution backend---which, as the evaluation shows, may be
unreliable. We have effectively \textbf{moved} the point of responsibility for correctness into the symbolic execution
tool. This has merit: A single, isolated source of definitions for instruction semantics must be implemented once
correctly and reliably to validate any other binary translator. Formal verification methods may be able to accomplish
this, after which Focaccia can apply their results to any other binary translation tool.

A clear deficiency of our method is that it is not systematic with respect to program inputs: It works on specific
instantiations of test cases, i.e., executions of programs. For each run, each executed instruction is tested for
exactly one concrete operand value. Generation of test cases must be systematic to maximize the verification's
exhaustiveness.
\\

Focaccia's source code resides at
\href{https://github.com/TUM-DSE/focaccia}{\texttt{https://github.com/TUM-DSE/focaccia}}.
