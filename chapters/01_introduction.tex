% !TeX root = ../main.tex
% Add the above to each chapter to make compiling the PDF easier in some editors.

\chapter{Introduction}\label{chapter:introduction}

% Motivation
In the development process of a binary translator, Arancini, we observed that, as is the case with all large-scale
software projects, its growing complexity started to hinder maintainability and progress. A particular difficulty
specific to the problem domain of binary translators is the fact that both input- and output data are binary---that is,
non-human-readable data, which poses significant difficulties during debugging in the development cycle. The most
logical way of dealing with binary data is to process it programmatically: our initial hypothesis is that automated
testing tools could partially alleviate restrictions of human cognition and facilitate development efficiency.

% Necessity
Most of the current software testing approaches, from unit testing to formal verification, rely heavily on manual work,
which is time consuming and prone to mistakes.

% Overview - what is Focaccia and how does it work?
This thesis designs and implements an automatic verifier for emulators: \textit{Focaccia}. It traces an emulated program
execution at instruction-level granularity and predicts for each instruction the program state that it \textbf{should}
produce based on the current emulator state and compares it to the state the emulator actually produces, thereby
establishing whether the emulator's implementation of an instruction acts on program state in accordance with its
specified semantics. To predict program states, Focaccia implements an oracle by harnessing concolic execution.
Additionally, this verifier is designed to be practical and usable with minimal setup work; specifically, it is a
standalone tool that does not require tight integration into an emulator's code base.
