% !TeX root = ../main.tex
% Add the above to each chapter to make compiling the PDF easier in some editors.

\chapter{Introduction}\label{chapter:introduction}

\section{Motivation}

In the development process of a binary translator, Arancini, we observed that, as is the case with all large-scale
software projects, its growing complexity started to hinder maintainability and progress. A particular difficulty
specific to an emulator's problem domain is the fact that both input- and output data are binary---that is,
non-human-readable data, which poses significant difficulties during debugging in the development cycle. The most
logical way of dealing with binary data is to process it programmatically: our initial hypothesis is that automated
testing tools could partially alleviate these restrictions of human cognition and facilitate development efficiency.

\section{Focaccia}\label{sec:intro:focaccia}

Focaccia verifies emulator correctness automatically. It traces an emulator's state at instruction-level granularity
during program execution and compares it to an oracle-based truth state, thereby establishing whether instruction
implementations act on program state in accordance with their specified semantics. Focaccia computes these test truths
itself by utilizing symbolic execution technology to implement said oracle.

\section{Challenges}\label{sec:intro:challenges}

List of stuff we do


Think about the ordering of sections in Design
