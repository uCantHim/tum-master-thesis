% !TeX root = ../main.tex
% Add the above to each chapter to make compiling the PDF easier in some editors.

\chapter{Introduction}\label{chapter:introduction}

\section{Motivation}

Arancini~\cite{arancini?}, a binary emulator, is a project at the \ac{DSE}. In the process of its development we
observed that, as is the case with all large-scale software projects, its growing complexity started to hinder
maintainability and progress. A particular difficulty specific to an emulator's problem domain is the fact that both
input- and output data are binary---that is, non human-readable data, which poses significant difficulties during
debugging in the development cycle. The most logical way of dealing with binary data is to process it programmatically:
our initial hypothesis is that automated testing tools could, in part, alleviate these restrictions of human cognition
and facilitate development efficiency.

\section{Approach}

Emulators or binary translators take as input machine instructions of an \ac{ISA} $X$ and produce semantically
equivalent instructions for an \ac{ISA} $Y$. In the special case of emulators, we call $X$ the \textit{guest} \ac{ISA}
and $Y$ the \textit{host} \ac{ISA}. Testing a translator (with respect to its correctness) means to check whether the
output it generates is in fact semantically equivalent to its input.

Formal verification, if indeed the highest attainable correctness standard, is often impractical for large-scale
projects. Thus, we employ a classical pragmatic testing paradigm in which manually compiled lists of test cases are
checked for correctness; one obtains a \textit{heuristic certainty}.

\section{Definitions}

% Definition 'Program'
A program $P_X = (x_i)_{1 \leq i \leq n}$ of an \ac{ISA} $X$ is a series of $n$ instructions, each of which is a mapping
of program states $S_i \mapsto x_i(S_i)$. Thus, a program can be expressed as
    $P_X : S_1 \mapsto S_{n+1} = x_n(S_n)$
with an arbitrary fixed input state $S_1$. Two programs are \textbf{semantically equivalent} if $P_X(S) = P_Y(S)$. In
this case, we write $P_X \equiv P_Y$.

% Definition 'translator', local discreteness
A binary translator (which may be an emulator) can be modelled as a function $E_{X \rightarrow Y}: P_X \mapsto P_Y$. It
is \textbf{correct} if the output program is semantically equivalent to the input program:
    $P_X \equiv E_{X \rightarrow Y}(P_X)$.
A binary translator is \textbf{locally discrete}, or context-independent, if it achieves $P_X \mapsto P_Y$ by mapping
each guest instruction individually to a semantically equivalent host program:
\begin{equation}\label{sem_eq}
    E_{X\rightarrow Y}(P_X) = E_{X \rightarrow Y}((x_i)_{1 \leq i \leq n}) = (E_{X \rightarrow Y}(x_i))_{1 \leq i \leq n} = P_Y
\end{equation}
with $E_{X\rightarrow Y}(x_i) \in \mathcal{P}_Y$ (where $\mathcal{P}_Y$ is the set of all programs of $Y$). In other
words, $E_{X\rightarrow Y}$ is distributive with respect to function application in $X$.

% Result local equivalence => global equivalence
If the semantic equivalence condition holds for each translation unit individually, it follows that it also holds globally:
\begin{equation}
    x_i \equiv E_{X\rightarrow Y}(x_i) \forall i \implies (x_i) \equiv (E_{X\rightarrow Y}(x_i)) \implies P_X \equiv P_Y
\end{equation}

\section{Focaccia}

In practice, binary translators are implemented as locally discrete translators. Focaccia exploits this property and
therefore takes on the endeavour of implementing the equivalence check $x \equiv E_{X \rightarrow Y}(x)$. Attempting to
solve this directly bears the following problems:

\begin{enumerate}
    \item What is $x$? Usually, an \ac{ISA} does not have a canonical way to represent its instructions as manipulable
        functions.
    \item What is $E_{X \rightarrow Y}(x)$? Translators, and particularly emulators, may be unable to communicate the
        translation they produce. The inability to generate it in the form of manipulable equations is universal.
    \item When is $x \equiv y$? A theorem checker could solve this problem.
    \item Comparing $x \equiv y$ is not enough, even if it were feasible, because the translation $E_{X \rightarrow
        Y}(x) = (y_j)_{1 \leq j \leq m} \in \mathcal{P}_Y$ may be a series of \textbf{multiple} instructions in $Y$, so
        the true question expands to: When is $x \equiv (y_j)$?
\end{enumerate}

Even though theorem solvers exist that can tackle problems 3 and 4 (for example Microsoft's Z3 theorem
prover~\cite{Z3prover2024Mar}), problem 2 makes this direct approach impossible. We can get around this by testing
whether $x(S) = [E_{X \rightarrow Y}(x)](S)$, which by definition implies $x \equiv E_{X \rightarrow Y}(x)$. The set of
problems to solve reduces to:

\begin{enumerate}
    \item What is $S = x(S_0)$? We call $S$ the \textit{truth state}.
    \item What is $S^E = [E_{X \rightarrow Y}(x_i)](S_0)$? We call the translator's output $S^E$ the \textit{test state}.
    \item When is $S = S^E$? As program states $S$ are constants, their equality is theoretically trivial. Practically,
        however, the start states of the guest program and the translator's output program are often not precisely
        equal. Therefore we implement an equivalence operator $S \equiv S^E$ which represents an equality with respect
        to differing start states. Chapter~\ref{chapter:implementation} discusses this necessity in detail.
\end{enumerate}
