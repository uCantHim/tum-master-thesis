\chapter{Implementation}\label{chapter:implementation}

\section{The Algorithm: An Abstract Description}

\subsection{Local Discreteness}\label{sec:impl:local_discreteness}

Requiring a tested translator to satisfy local discreteness is imperative to verify whether $P_X \equiv E_{X \rightarrow
Y}(P_X)$. This is because, if this requirement weren't assumed, a verifier would either have to hard-code the result of
comparing every possible program of $X$ to every possible program of $Y$ (which is obvious nonsense as both of these
sets are infinitely large) or be another sort of `magical' algorithm that is not conceivable by human reason. This
argument simultaneously shows that every implementation of a binary translator \textbf{must necessarily} be locally
discrete for the same reasons.

However, when we use the definition of program equivalence to transform the problem into $P_X(S) = [E_{X \rightarrow
Y}(P_X)](S)$, the global comparison of result states becomes possible again. While this removes the theoretical
necessity for the local discreteness assumption, it is still practically useful, if not essential. An outcome of
comparing result states of full program executions would be a statement like "The program's translation is erroneous".
This is too general of an assertion to be useful to developers. When we instead use local discreteness and check for all
intermediate states whether $S_i = S^E_i$ (where $S_i = x_{i-1}(S_{i-1})$ and $S^E_i = E_{X \rightarrow
Y}(x_{i-1})(S^E_{i-1})$ with $S_1 = S^E_1$ an arbitrary initial state), we can generate much more detailed information,
such as "Instruction $x_i$ is translated incorrectly because $S_i \neq S^E_i$". This is exactly the type of feedback
that Focaccia seeks to provide.

Additionally, it lets the algorithm recover from errors: Even if an instruction $x_i$ turns out to be implemented
incorrectly, thus producing an incorrect state $S^E_{i+1}$, that state is only incorrect from the viewpoint of program
semantics. The verifier, however, can still treat it as a valid input state to all subsequent instructions $x_{j > i}$
because the translation $E_{X \rightarrow Y}(x_j)$ is required to be locally correct for a locally discrete translator,
that is, it must work correctly on \textbf{any} program state.

\subsection{Comparing Program States}\label{sec:impl:comparison}

A naive verification algorithm works as follows: We run $P_X$ on one initial state $S$ and the translation $E_{X
\rightarrow Y}(P_X)$ on the same initial state, during which we record the respective intermediate states $S_i$ and
$S^E_i$. For each pair of states, test whether $S_i = S^E_i$. If this equality does \textbf{not} hold, then the
translator's implementation of $x_i$, meaning the translation $x_i \mapsto (y_j)$, is faulty.

The intuitive way of implementing the equality operator on program states ($S = S'$) is by comparing register- and
memory content. These values are what constitute the state of a program, and they are numeric values with a canonical
condition for equality. However, as section \ref{sec:intro:focaccia} indicated, comparing program states is more complex
in reality. That is because the starting states for the execution of programs are not always equal: the previous
assumption that $S_1 = S^E_1$ is not necessarily true. In fact, it is most commonly false. Possible contributing factors
(subsequently collectivized as a general \textit{difference in environment}) include:

\begin{enumerate}
    \item Different initial stack pointers.
    \item Different addresses of heap allocations.
    \item Different environment- and auxiliary vectors. The latter is particularly interesting. It turns out that
        the auxiliary vector provided by QEMU, for example, routinely differs from the one provided by the operating
        system. See section \ref{sec:auxv} for more details.
\end{enumerate}

Therefore, instead of comparing $P_X(S) = P_Y(S')$, which does \textbf{not} imply $P_X \equiv P_Y$ if $S \neq S'$, the
comparison must take into account the initial difference $\Delta_S = S - S'$ and establish a \textbf{state equivalence}
$P_X(S) \equiv_{\Delta_S} P_Y(S') \implies P_X \equiv P_Y$ with respect to it. This is the chief nontriviality that
Focaccia's algorithm solves.

The way we calculate this equivalence is by re-introducing information about the guest instructions $x_i$ to the
algorithm which, in order to simplify the problem, we have discarded when we transformed the central question from $x
\equiv E_{X \rightarrow Y}(x)$ to $x(S) = [E_{X \rightarrow Y}(x)](S)$. The new algorithm works as follows: Instead of
running guest program and translation in parallel and comparing their intermediate states, only run the translation
$E_{X \rightarrow Y}(P_X)$ on a start state $S^E_1$, thereby obtaining the translation's intermediate states $S^E_i$.
Then, for each $S^E_i$, use the guest instruction $x_i$ to calculate a corresponding \textbf{expected state} $S_{i+1} =
x_i(S^E_i)$. These represent truth states that would result from executing $x_i$ on $S^E_i$ if $x_i$ was implemented
correctly. Finally, compare the expected state to the actual translation state: $S^E_{i+1} = S_{i+1}$. Again, this works
because it does not matter to the verifier whether $S^E_{i+1}$ is a \textit{correct} state with regards to whole-program
semantics.

Decomposing the pre-translation program $(x_i)$ into its instructions and applying them selectively to the synthetic
test states as opposed to executing it natively on a semi-random starting state thus allows us to use the same starting
state for both the translation and the truth program at each instruction, eliminating $\Delta_S$. This enables the
desired equality comparison $x_i(S_i) = E_{X \rightarrow Y}(x_i)(S_i)$. The drawback of this approach is that we now
require an additional piece of information: We need to know $x_i$.

Not only do we need to know what every $x_i$ is, but we also demand a way of applying it individually to an arbitrary
program state which we determine, or, more precisely, which is being determined by the translation's execution.

\section{Symbolic Execution}

\subsection{Obtaining the Instruction}

In the first prototype, we used a disassembly framework to load a binary and disassemble it in its entirety. This was
slow and produced lots of unnecessary computations, so we started disassembling instructions on demand, i.e. at each
program counter only read the next instruction. This works for statically linked binaries as all information is in one
file and can be loaded from there. In the third iteration, in order to support dynamically linked programs and even
\ac{JIT} compiled code, we omit loading the binary entirely and instead read the next instruction directly from the
running program's memory; this can be either a native execution (see \ref{sec:impl:concrete_exec}) or a running
translator/emulator.

\subsection{A Symbolic Representation}

Once we have an instruction's byte code, we need to apply it to a program state.

Essentially, we rely on one translator (in this case one which translates instructions into symbolic equations) to be
implemented correctly: It is an oracle.

\subsection{Verifiying the Symbolic Execution Backend}

\section{Concrete Guidance}\label{sec:impl:concrete_exec}

\section{Trace Mismatch}\label{sec:impl:trace_mismatch}

One can imagine code that behaves effectively as the following:

\begin{lstlisting}[language=Python]
    n = random()
    if n > 0.5:
        func_a()
    else:
        func_b()
\end{lstlisting}

These situations do happen, be it during iteration over environment arrays or their content, literally deciding branches
based on randomness, or generally any branching involving nondeterministic values (networking, time, system state, …). A
specific situation in which this routinely happens is the libc initialization code, where auxiliary and environment
vectors are processed. These nondeterministic mutations can cause the symbolic program trace differ from the tested
program trace.

Per-instruction symbolism cannot process these situations; the symbolic truth simply does not provide information on
instructions that were not part of its program execution. An algorithm <no local discreteness assumption could solve
this, but is not possible with symbolic exec: state explosion>

\textbf{NOTE} Can be eliminated for the special case of online emulator verification. We could implement a minimal
custom protocol as a replacement for the GDB-server protocol and provide a general online verification algorithm for
emulators that implement this protocol specifically for verification with Focaccia. This algorithm would give the
highest-quality results that Focaccia is able to calculate.



% APPROACH
%
%  - systematic!
%  - we don't want to write tests
%  - fuzzing has more setup work, is harder to use
%
% chapter 12 virtual machines popek & goldberg
%
% establishing V(S_j)
