\chapter{Implementation}\label{chapter:implementation}

Focaccia is a Python application under a strong `library-style code that is being tied together only by the user-facing
script' paradigm. Its object structure does not follow a strict Java-like class-oriented programming style, but instead
fits loose components (mostly free functions that act on well-defined data types) onto the underlying conceptual data
pipeline.

\section{Tools}

\subsection{Symbolic Execution Backend}

A symbolic execution backend, in order to be useful for our specific purpose, must satisfy multiple conditions:

\begin{itemize}
    \item Symbolic execution for assembly languages (many tools operate on high-level languages); primarily x86, but
        preferredly more
    \item Python API
    \item Freedom to execute single instructions symbolically on arbitrary states
    \item Low-level access to generated symbolic expressions
\end{itemize}

The first tool in consideration was the binary analysis framework \textit{angr}~\cite{shoshitaishvili2016state}. angr is
a mature and well-tested open source platform for control-flow graph recovery, symbolic execution, disassembly and
lifting, and more~\cite{AngrWebsite2024Mar}. Still, angr ended up being rejected for the use as Focaccia's symbolic
execution backend. Almost none of our tasks were easily accomplished with angr; the interface is so tuned to high-level
cyber security tasks that our attempts to accomplish `just this one simple thing' grew into a constant battle against
the API\@. Not only does angr bury its low-level functionality, which would have sufficed for Focaccia's purposes, under
huge stacks of abstraction, delegation, and opaqueness, but also turned out to be slow while doing that.

Further research has led us to adopt Miasm~\cite{desclaux2012miasm}, a reverse engineering framework developed by CEA IT
Security at the French Alternative Energies and Atomic Energy Commission. Its features include
opening/modifying/generating binary files, assembling/disassembling a host of assembly languages, emulation, symbolic
execution, and \textbf{representing assembly semantics using an intermediate language}~\cite{cea-sec2024Mar}. The latter
statement hints towards exactly the functionality that Focaccia requires.

As a proposal for future improvement of Focaccia, it is entirely possible to build a custom abstraction layer over the
symbolic execution backend as a means to have multiple interchangeable backends for feature- and implementation
completeness of all desired architectures. We discarded this idea initially for its likelihood to incur unnecessary
effort, though the evaluation has shown that the backend's completeness really is a major pain point of Focaccia.

\subsection{Concrete Execution Tracer}

We need to follow a program's native execution on a per-instruction basis and inspect the program's state at each trace
point (see section~\ref{sec:concolic_tracing}). A debugger is a canonical choice for this task; we use LLDB because it
has a Python API (one can start and manipulate debugger instances conveniently in Python code)~\cite{lldb2024Apr}.
\lstlistingname~\ref{fig:lldb_tracing} shows a minimal example of how LLDB can be used to trace a native execution.

\begin{figure}[htbp]
    \centering
    \begin{tabular}{c}
    \begin{lstlisting}[language=Python]
        import lldb
        debugger = lldb.SBDebugger.Create()
        debugger.SetAsync(False)
        target = debugger.CreateTargetWithFileAndArch("/bin/ls",
                                                      lldb.LLDB_ARCH_DEFAULT)
        error = lldb.SBError()
        process = target.Launch(debugger.GetListener(),
                                argv, envp, None, None, None,
                                None, 0, True,
                                error)

        while self.process.GetState() != lldb.eStateExited:
            thread = process.GetThreadAtIndex(0)
            frame = thread.GetFrameAtIndex(0)

            pc = frame.FindRegister("PC").GetValueAsUnsigned()
            instr = disassemble(process.ReadMemory(pc, 16, error))
            thread.StepInstruction()
    \end{lstlisting}
    \end{tabular}
    \caption{Tracing a concrete execution with LLDB}\label{fig:lldb_tracing}
\end{figure}

\section{Algorithms}\label{sec:algorithms}

\subsection{Generating Symbolic Expressions}\label{sec:symb_expr_impl}
