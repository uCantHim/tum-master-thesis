\chapter{Background}

\section{Symbolic Execution}

Symbolic execution is a program analysis technique originally developed for software testing in the mid '70s based on
previous advances in the field of formal verification~\cite{Boyer1975Select, King1975Effigy, Howden1977Dissect}. The
technique's proposition is to execute programs with symbolic values instead of concrete ones, and track their evolution
throughout the program together with conditions that branches impose on the existence of certain values. This way, a
program execution represents many to all possible values and/or paths at the same time. The results are symbolic
equations for the contents of variables or memory, which allow theorem solvers to reason about general properties of the
code. Statements like "Assertion X can never fail", "For all $Y > 5$, path A taken instead of path B", or "Statement Z
never dereferences a \texttt{nullptr}" can be proven with formal certainty.

\lstlistingname~\ref{fig:symbexec_example_listing} shows a code example. In a usual \textit{concrete} execution,
\texttt{a} is exactly one value and subsequent code uses that value to perform calculations. If, for example, we set
\texttt{a = 4}, the program calculates the following values: \texttt{a = 4}, \texttt{b = -1}, \texttt{c = -3}. If, on
the other hand, in a symbolic execution of the same code, \texttt{a} is set to a generic symbol $\alpha$, a symbolic
execution engine produces the following output: \texttt{a = $\alpha$}, \texttt{b = $\alpha$ - 5}, \texttt{c =
((($\alpha$ - 5) \% 3) == 0) ? ($\alpha$ + 10) : (-3)}. The entire program is expressed depending on the generic symbol
$\alpha$.

\begin{figure}[htbp]
    \centering
    \begin{tabular}{c}
    \begin{lstlisting}[language=Python]
        a = ?
        b = a - 5
        if (b % 3) == 0:
            c = a + 10
        else:
            c = -3
    \end{lstlisting}
    \end{tabular}
    \caption[Symbolic execution example]{Symbolic Execution Sample Code}\label{fig:symbexec_example_listing}
\end{figure}

However, the theoretical vision of perfectly formal symbolic execution turned out to be hardly feasible in practice.
King, in one of the first papers describing symbolic execution as a novel formal verification technique, noted in 1976:
"Many of the troublesome issues arising in program proving systems also occur in symbolic
execution"~\cite{King1976SymbExec}. The problems he mentioned include array access based on a symbolic index,
finite-precision of floating point numbers which inhibits mathematically sound, but practically invalid transformations
of symbolic equations for the sake of simplification (e.g., floating-point addition is not generally commutative on
computer hardware), and theoretical limitations in decidability for theorem provers. The situation has not improved
since then---if anything, \textbf{more} problems have been uncovered, especially as the complexity of programming
languages to which symbolic execution is to be applied has increased massively.

Despite the current problems in state-of-the-art symbolic execution tools like state/path explosion on real nontrivial
programs, finding appropriate models for symbolic memory (similar to King's array dereferencing problem), and side
effects of library and system code, symbolic execution can still be used effectively when applied to reasonably
restricted scenarios or when enhanced with advanced heuristic approaches~\cite{Baldoni2018SymbexecSurvey}.
\textit{Concolic execution}~\cite{Sen+2005Cute, Sen2007ConcolicTesting}, in which symbolic execution is selectively
guided by concrete state to avoid combinatory explosions while still achieving reasonably exhaustive results, is one of
these more recent developments. Focaccia's algorithm is a customized variant of this concept.

% "Symbolic execution is a popular program analysis technique introduced in the mid ’70s to test whether certain
% properties can be violated by a piece of software [16, 58, 67, 68]"~\cite{Baldoni2018SymbexecSurvey}.
%
%     -> https://dl.acm.org/doi/10.1145/390016.808445 (1975) [Boyer+ SELECT: refers to King, proposes a concrete tool]
%     -> https://ieeexplore.ieee.org/document/1702443 (1977) [Howden DISSECT: cites Boyer+]
%     -> https://dl.acm.org/doi/10.1145/390016.808444 (1975) [King EFFIGY: language seems to coin the term "symbolic execution"]
%     -> https://dl.acm.org/doi/10.1145/360248.360252 (1976) [King: cites King, Boyer+, abstract description of SE]
%
% First tools were proposed in April 1975 by both Boyer+ and King, seemingly independent of each other. King elaborates on
% the theoretical properties of symbolic execution in a 1976 follow-up paper that also cites Boyer+.

\section{Binary Translation and Emulation}

The idea of binary translation arises when one considers the reality of multiple conflicting instruction set
architectures in the world and the desire or even necessity to execute programs on all of them. The natural way to
accomplish this is to compile a program for each of the targeted architectures, resulting in \textit{native} executables
for each, which also makes this approach generate the most efficient form of program execution as it incurs no runtime
overhead at all. This works for entirely self-contained programs, but programs tend to depend upon third-party software
of which the source code might not be available at all or may perform architecture-specific work so that it is not
trivially adaptable to different architectures. Additionally, recompilation and the distribution of multiple binaries
associated with it is inconvenient and also costly for very large projects.

Other approaches, which trade off different parts of the aforementioned pros and cons, can be assigned to the rough
categories of microcoded emulators, software interpreters, and binary translators~\cite{sites1993binary}. The first one
is concerned with supporting multiple \ac{ISA}s \textbf{in hardware} by translating their instructions to a common
microcode language; this issue shall not trouble us here. The latter two are software solutions, also known as dynamic
binary translation and static binary translation, respectively~\cite{cifuentes1996staticdynamic, Rocha2022Lasagne}.
Static binary translation takes as input binary code of one \ac{ISA} and compiles it to a different \ac{ISA} in an
offline process ahead of time. It may include a runtime component that maps operating system calls to the target
platform. Static translation cannot deal self-modifying code, but incurs little runtime overhead. Dynamic binary
translation, on the other hand, acts in the spirit of an interpreter by translating a binary on the fly during its
execution. It handles dynamic branching without problems, but its runtime overhead is higher because of the continually
running interpreter, and optimization options for generated code are severely limited by its narrow, local view of the
executed code, even though many optimization techniques have been developed over the
years~\cite{Guan+2010DbtOptimizations, Sun+2023DbtBranchPred, Hawkins2015OptimizingDbt, Kedia+2013DbtKernel}. Programs
that perform dynamic binary translation are frequently called \textit{emulators}.
