\chapter{Background}

\section{Symbolic Execution}

Symbolic execution is a program analysis technique originally developed for software testing in the mid '70s based on
previous advances in the field of formal verification~\cite{Boyer1975Select, King1975Effigy, Howden1977Dissect}. The
technique's proposition is to execute programs with symbolic values instead of concrete ones, and track their evolution
throughout the program as well as conditions imposed on the existence of certain values by branches. This way, a program
execution represents many to all possible values and/or paths at the same time. The results are symbolic equations for
the contents of variables or memory, which allow theorem solvers to reason about general properties of the code.
Questions like "Can assertion X ever fail?", "For which values of variable Y is path A taken instead of path B?", or
"Does statement Z ever dereference a \texttt{nullptr}?" can be answered with formal certainty.

However, the theoretical vision of perfectly formal symbolic execution turned out to be hardly feasible in practice.
King, in one of the first papers describing symbolic execution as a novel formal verification technique, noted in 1976:
"Many of the troublesome issues arising in program proving systems also occur in symbolic
execution"~\cite{King1976SymbExec}. The problems he mentioned include array access based on a symbolic index,
finite-precision of floating point numbers which inhibits mathematically sound, but practically invalid transformations
of symbolic equations for the sake of simplification (e.g., floating-point addition is not generally commutative on
computer hardware), and theoretical limitations in decidability for theorem provers. The situation has not improved
since then---if anything, \textbf{more} problems have been uncovered, especially as the complexity of programming
languages to which symbolic execution is to be applied has increased massively.

Despite the current problems in state-of-the-art symbolic execution tools like state/path explosion on real nontrivial
programs, finding appropriate models for symbolic memory (similar to King's array dereferencing problem), and side
effects of library and system code, symbolic execution can still be used effectively when applied to reasonably
restricted scenarios or when enhanced with advanced heuristical approaches~\cite{Baldoni2018SymbexecSurvey}.
\textit{Concolic execution}~\cite{Sen+2005Cute, Sen2007ConcolicTesting}, in which symbolic execution is selectively
guided by concrete state to avoid combinatory explosions while still achieving reasonably exhaustive results, is one of
these more recent developments. Focaccia's algorithm is a customized variant of this concept.

% "Symbolic execution is a popular program analysis technique introduced in the mid ’70s to test whether certain
% properties can be violated by a piece of software [16, 58, 67, 68]"~\cite{Baldoni2018SymbexecSurvey}.
%
%     -> https://dl.acm.org/doi/10.1145/390016.808445 (1975) [Boyer+ SELECT: refers to King, proposes a concrete tool]
%     -> https://ieeexplore.ieee.org/document/1702443 (1977) [Howden DISSECT: cites Boyer+]
%     -> https://dl.acm.org/doi/10.1145/390016.808444 (1975) [King EFFIGY: language seems to coin the term "symbolic execution"]
%     -> https://dl.acm.org/doi/10.1145/360248.360252 (1976) [King: cites King, Boyer+, abstract description of SE]
%
% First tools were proposed in April 1975 by both Boyer+ and King, seemingly independent of each other. King elaborates on
% the theoretical properties of symbolic execution in a 1976 follow-up paper that also cites Boyer+.

\section{Binary Translation and Emulation}
