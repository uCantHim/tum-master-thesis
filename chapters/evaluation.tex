\chapter{Evaluation}

The primary evaluation target for Focaccia is QEMU with x86 as the guest architecture. Ideally, Focaccia would also be
tested with Arancini, which was the initial motivation for this project. Unfortunately, support for most of the targets
Arancini aims to support is not polished enough to be able to execute full programs at the time of writing.
Additionally, its logging capabilities are severely limited and restrict the calculations that Focaccia is able to
perform to a significant degree.

To test Focaccia's functionality, we attempt to reproduce and detect bugs that were already reported, confirmed, and
fixed so that we have a verifiable measure for success. A search on the gitlab issue tracker yields a total of 235 bug
reports with a \texttt{target:i386} tag, nine of which are instruction semantics related bugs. It is noteworthy how rare
this specific type of bug on which Focaccia specializes is in mature implementations.

\section{Setup}

For each bug, we compile the newest version of QEMU that still contained the bug. We try to reproduce the bug with a
minimal program, which is usually included in the bug report. If reproduction succeeds, we run Focaccia on the emulator
with the same program and see if it detects the bug automatically.

Not much is to be said about the concrete testing process because using Focaccia does not incur a lot of effort: We
generate a symbolic trace, then start a QEMU instance with the GDB-stub active to provide an interface for Focaccia, and
finally execute the verifier script that interfaces with a GDB server, feeding it the pre-computed symbolic oracle.
\figurename~\ref{fig:testing_process} shows an example, where \texttt{bug.out} is the minimal reproducing program. Both
Python programs that are being invoked are utilities provided by Focaccia.

\begin{figure}[htbp]
    \centering
    \begin{tabular}{c}
    \begin{lstlisting}[language=bash]
        python capture_transforms.py -o oracle.trace bug.out
        qemu-x86_64 -g 12345 bug.out &
        python verify_qemu.py --symb-trace oracle.trace localhost 12345
    \end{lstlisting}
    \end{tabular}
    \caption{Testing Process}\label{fig:testing_process}
\end{figure}

\section{Results}

By performing the process described above for all nine bugs, we reach the following results:

\begin{itemize}
    \item 6 bugs were not detectable because the faulty instruction is not implemented by the symbolic execution
        backend. In this case, Focaccia prints a warning and skips the instruction.
    \item 1 bug was not detectable because the faulty instruction's malfunction caused QEMU to crash before Focaccia was
        able to read its state.
    \item 1 bug was detected successfully.
    \item We were unable to reproduce one of the bugs entirely.
\end{itemize}

These results showcase the impact that reliance on a correct and complete symbolic execution backend has in a real-world
scenario. The unimplemented instructions are so rare that errors in their implementation were only found and reported in
2021, many years into both QEMU's and Miasm's widespread production use. However, a ratio of $\frac{1}{8}$ reproducible
bugs detected is still lower than expected and indicates that the assumption of higher implementation correctness of
symbolic execution tools than emulators might be unreasonable---at least if one aims to attain a deeper than superficial
attestation of correctness. Possible actions are discussed in chapter~\ref{chapter:future_work}.

\subsubsection{Details}

The unimplemented instructions are \texttt{ADDSUBPS} (SSE), \texttt{BZHI}, \texttt{BEXTR}, \texttt{BLSMSK},
\texttt{BLSI/BLSR}, and \texttt{VPSHUFB} (AVX, which is entirely unsupported by the SE backend).

The crashed case was the instruction \texttt{ADOX}. Note that the crash only happened when QEMU was executed with the
\texttt{-g <port>} flag that enables its \texttt{gdbserver} interface, which is required for verification with Focaccia.
This seems to have been another bug in QEMU that is not directly related to emulation.

The bug that was successfully found by Focaccia is in the implementation of \texttt{CMPXCHG}, which was reported in
gitlab issue \href{https://gitlab.com/qemu-project/qemu/-/issues/508}{\#508}. QEMU zero-extends \texttt{EAX} to
\texttt{RAX}, even though it should not if the compared values are equal: In the test case, \texttt{RAX} should contain
the unmodified value \texttt{0x1234567812345678} after \texttt{CMPXCHG} is executed, but QEMU sets it to
\texttt{0x0000000012345678}. Focaccia's output to that instruction is shown in \lstlistingname~\ref{fig:cmpxchg}
(wrapped to fit on the page).

\begin{figure}[htb]
    \centering
    \begin{tabular}{c}
    \begin{lstlisting}
----------------------------------------------------------------------
For PC=0x40116a
----------------------------------------------------------------------
  1. [ERROR] Content of register RAX is false.
     Expected value: 0x1234567812345678,
     actual value in the translation: 0x12345678.

Expected transformation: Symbolic state transformation 0x40116a -> 0x40116e:
  [Symbols]
    ZF     = (-@32[RBP + 0xFFFFFFFFFFFFFFEC]) == (-RAX[0:32])
    RAX    = ((-@32[RBP + 0xFFFFFFFFFFFFFFEC]) == (-RAX[0:32]))
             ?(RAX,{@32[RBP + 0xFFFFFFFFFFFFFFEC] 0 32, 0x0 32 64})
    RIP    = 0x40116E
    @32[RBP + 0xFFFFFFFFFFFFFFEC]
           = ((-@32[RBP + 0xFFFFFFFFFFFFFFEC]) == (-RAX[0:32]))
             ?(RDI[0:32],@32[RBP + 0xFFFFFFFFFFFFFFEC])
  [Instructions]
    CMPXCHG    DWORD PTR [RBP + 0xFFFFFFFFFFFFFFEC], EDI
Actual difference: Snapshot (x86_64):
                   {'RBP': '0x0', 'RAX': '0xedcba98800000000', 'RIP': '0x4'}
    \end{lstlisting}
    \end{tabular}
    \caption{Verifier Output for Faulty \texttt{CMPXCHG} Instruction}\label{fig:cmpxchg}
\end{figure}
