\chapter{Evaluation}

\section{QEMU}

For QEMU with an x86 guest as the primary evaluation target, we attempt to reproduce and detect bugs that were reported
and fixed in the past. A search on the gitlab issue tracker yielded a total of 9 instruction-semantic related bug
reports out of a total of 235 bug reports with a \texttt{target:i386} tag. It is notable how rare the specific type of
bug on which Focaccia specializes is in mature emulators.

We use minimal programs to reproduce these bugs, then run Focaccia on a version of QEMU that still contained the
respective bug to see whether Focaccia can detect it. Not much more is to say about the methodology and setup as
Focaccia facilitates the entire workflow out of the box: We only start a QEMU instance and execute one of Focaccia's
scripts that connects to that instance and verifies it. These are the results:

\begin{itemize}
    \item 6 bugs were not detectable because the faulty instruction is not implemented by the symbolic execution
        backend. In this case, Focaccia prints a warning and skips the instruction.
    \item 1 bug was not detectable because the faulty instruction's malfunction caused QEMU to crash before Focaccia was
        able to read its state.
    \item 1 bug was detected successfully.
    \item We were unable to reproduce one of the bugs entirely.
\end{itemize}

These results showcase the impact that reliance on a correct and complete symbolic execution backend has in a real-world
scenario. The unimplemented instructions are so rare that errors in their implementation were only found and reported by
a research project at KAIST in 2021, many years into both QEMU's and Miasm's widespread production use. However, a ratio
of $\frac{1}{8}$ reproducible bugs detected is still lower than expected and indicates that the assumption of higher
implementation correctness of symbolic execution tools than emulators might be unreasonable---at least if one aims to
attain a deeper than superficial attestation of correctness.

\subsubsection{Details}

The unimplemented instructions are \texttt{ADDSUBPS} (SSE), \texttt{BZHI}, \texttt{BEXTR}, \texttt{BLSMSK},
\texttt{BLSI/BLSR}, and \texttt{VPSHUFB} (AVX, which is entirely unsupported by the SE backend).

The crashed case was the instruction \texttt{ADOX}. Note that the crash only happened when QEMU was executed with the
\texttt{-g <port>} flag that enables its \texttt{gdbserver} interface, which is required for verification with Focaccia.
This seems to have been another bug in QEMU that is not directly related to emulation.

The bug that was successfully found by Focaccia is in the implementation of \texttt{CMPXCHG}, which was reported in
gitlab issue \href{https://gitlab.com/qemu-project/qemu/-/issues/508}{\#508}. QEMU zero-extends \texttt{EAX} to
\texttt{RAX}, even though it should not if the compared values are equal: In the test case, \texttt{RAX} should contain
the unmodified value \texttt{0x1234567812345678} after \texttt{CMPXCHG} is executed, but QEMU sets it to
\texttt{0x0000000012345678}. Focaccia's output to that instruction is shown in \lstlistingname~\ref{fig:cmpxchg}
(wrapped to fit on the page).

\begin{figure}[htbp]
    \centering
    \begin{tabular}{c}
    \begin{lstlisting}
----------------------------------------------------------------------
For PC=0x40116a
----------------------------------------------------------------------
  1. [ERROR] Content of register RAX is false.
     Expected value: 0x1234567812345678,
     actual value in the translation: 0x12345678.

Expected transformation: Symbolic state transformation 0x40116a -> 0x40116e:
  [Symbols]
    ZF     = (-@32[RBP + 0xFFFFFFFFFFFFFFEC]) == (-RAX[0:32])
    RAX    = ((-@32[RBP + 0xFFFFFFFFFFFFFFEC]) == (-RAX[0:32]))
             ?(RAX,{@32[RBP + 0xFFFFFFFFFFFFFFEC] 0 32, 0x0 32 64})
    RIP    = 0x40116E
    @32[RBP + 0xFFFFFFFFFFFFFFEC]
           = ((-@32[RBP + 0xFFFFFFFFFFFFFFEC]) == (-RAX[0:32]))
             ?(RDI[0:32],@32[RBP + 0xFFFFFFFFFFFFFFEC])
  [Instructions]
    CMPXCHG    DWORD PTR [RBP + 0xFFFFFFFFFFFFFFEC], EDI
Actual difference: Snapshot (x86_64):
                   {'RBP': '0x0', 'RAX': '0xedcba98800000000', 'RIP': '0x4'}
    \end{lstlisting}
    \end{tabular}
    \caption{Verifier Output for Faulty \texttt{CMPXCHG} Instruction}\label{fig:cmpxchg}
\end{figure}

\section{Arancini}

Support for most of the targets Arancini aims to support is currently not polished enough to be able to execute full
programs. Additionally, logging is limited and severely restricts the calculations that Focaccia is able to perform;
these statements are, of course, results in themselves.
